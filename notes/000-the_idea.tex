\section{想法的产生}

围棋是一种复杂的棋类游戏,为了攻克它,人类把所有的认知工具都搬了过来,比如,棋感是一种直觉,棋招的计算是决策过程中对解空间的搜索。
最有趣的事情是,人们在这个过程中,创造了一系列的概念来辅助下棋,比如实地、外势、厚、薄等等。这些概念因为来自于直观,有它的模糊性。

最近 WeiqiTV 和 Deepmind 对 AlphaGo 的五盘自战棋做了详细解读,让人可以一窥它高深的棋力。在这个节目的解说中,樊麾提到 AlphaGo
对中腹有独到的认识。而常昊也说,人类能够计算清楚角和边,但算不清楚腹地。在下棋的过程中,人类往往会回避掉下在腹地的可能,因为计算困难,
妨碍了人们行棋的自由。

一个想法是:AI 能否和人类一起去探索围棋之道呢?它能否帮助人们创造性的提出一些围棋的新概念呢?这些概念可用来推理,并辅助人们来决策复杂问题。
这要求我们能够有一个基础来定义围棋里的概念,而这个让我想起了 Hoare Logic 。










