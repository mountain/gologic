

围棋是一种复杂的棋类游戏,人们为了攻克它把我们认知工具所有的武器都搬了过来,比如,棋感是一种直觉,计算是决策过程中对解空间的搜索。
而最有趣的事是人们在这个过程中创造了一系列的概念来辅助我们下棋,比如实地、外势、厚、薄等等。这些概念因为来自于直观,有它的模糊性。

最近 WeiqiTV 和 Deepmind 的樊麾对 AlphaGo 的五盘自战棋做了详细解读,让人可以一窥它高深的棋力。在这个节目的解说中,樊麾提到 AlphaGo
对中腹有独到的认识。而常昊也说到,人类因为能够计算清楚角和边,而算不清楚腹地,导致在下棋的过程中回避掉算不清楚带来的不确定性。
这种因为困难带来的无知境地,妨碍了人们行棋的自由。

一个想法是:AI 能否和人类一起去探索围棋之道呢?它能否帮助人们创造性的提出一些围棋的新概念呢?这些概念可用来推理,并辅助人们来决策复杂问题。

这要求我们能够有一个基础来定义围棋里的概念,而这个让我想起了 Hoare Logic 。








