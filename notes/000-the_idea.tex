\section{想法的产生}

围棋是一种复杂的棋类游戏,为了攻克它,人类把所有的认知工具都搬了过来,比如,棋感是一种直觉,棋招的计算是决策过程中对解空间的搜索。
最有趣的事情是,人们在这个过程中,创造了一系列的概念来辅助下棋,比如实地、外势、厚、薄等等。这些概念因为来自于直观,有它的模糊性。

最近 WeiqiTV 和 Deepmind 对 AlphaGo 的五盘自战棋做了详细解读,让人可以一窥它高深的棋力。在这个节目的解说中,樊麾提到 AlphaGo
对中腹有独到的认识。而常昊也说,人类能够计算清楚角和边,但算不清楚腹地。在下棋的过程中,人类往往会回避掉下在腹地的可能,因为计算困难,
妨碍了人们行棋的自由。

一个想法是:AI 能否和人类一起去探索围棋之道呢?它能否帮助人们创造性的提出一些围棋的新概念呢?这些概念可用来推理,并辅助人们来决策复杂问题。
这要求我们能够有一个基础来定义围棋里的概念,而这个让我想起了 Hoare 逻辑 。

\subsection{围棋机器的定义}

在引入 Hoare Logic 之前,在这一节首先让我们引入围棋机器,我们会给出它的硬件和操作系统的描述。

\begin{center}
\begin{psgoboard}
\move*{q}{16}
\move*{d}{16}
\end{psgoboard}
\end{center}

让我们从棋盘上格点的表示开始。如上,$\mathbf{q16}$ 和 $\mathbf{d16}$ 分别是黑方和白方所下的两个位置。整个棋盘有 361 个格点位置,
这对应于围棋机器的核心部件—盘面寄存器阵列。它由361个格点寄存器构成。每个格点寄存器的标号,就是相应格点在棋盘上面的标号。

每个格点寄存器有三种状态:$\mathit{black}$ 、 $\mathit{white}$ 和 $\mathit{unknown}$。

下棋的过程中,黑方和白方轮流进行,这对应于一个轮次寄存器 $\mathbf{turn}$,它有三种状态:$\mathit{black}$ 、 $\mathit{white}$、 $\mathit{judge}$。

围棋机器有指令寄存器,它的状态是指令集中的一条指令。指令集是由所有落子的可能构成,显然它们也是盘面上的所有位置,但我们用大写字母来标注它们。
另外,还有一条特殊的指令是 $\mathbf{PASS}$。

\setcounter{gomove}{0}
\begin{center}
    \begin{psgoboard}
        \move*{q}{16}
        \move*{d}{16}
        \move*{r}{4}
        \move*{d}{4}
    \end{psgoboard}
\end{center}

围棋机器里有三个常驻内存程序,它们具有类似的形式,但语义略有不同

\begin{itemize}
    \item 裁判子程序 $\mathbb{J}$:读入盘面寄存器的状态,给出一条指令,是期望去指定位置提子
    \item 黑方子程序 $\mathbb{B}$:读入盘面寄存器的状态,给出一条指令,是期望去指定位置下黑子
    \item 白方子程序 $\mathbb{W}$:读入盘面寄存器的状态,给出一条指令,是期望去指定位置下白子
\end{itemize}

围棋机器有一个非常简单的操作系统,它轮流执行上面三个子程序,根据轮次和指令,分别去提子、下黑子、下白子,或者报出错误。它对应于一个状态迁移表。

如上的开局作战,对应的指令执行序列是:$\mathbf{Q16; PASS; D16; PASS; R04; PASS; D04}$


\subsection{Hoare逻辑和程序规范}
















