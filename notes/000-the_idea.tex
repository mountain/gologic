\documentclass[a4paper,10.5pt]{article}

\usepackage[top=1in,bottom=1in,left=1in,right=1in]{geometry} % 用于设置页面布局
\usepackage{xeCJK} % 用于使用本地字体
\usepackage[super, square, sort&compress]{natbib} % 处理参考文献
\usepackage{titlesec, titletoc} % 设置章节标题及页眉页脚
%\usepackage{xCJKnumb} % 中英文数字转换
\usepackage{amssymb}
\usepackage{amsmath} % 在公式中用\text{文本}输入中文
\usepackage{diagbox}
\usepackage{multirow} % 表格中使用多行
\usepackage{booktabs} % 表格中使用\toprule等命令
\usepackage{rotating} % 使用sidewaystable环境旋转表格
\usepackage{tabularx}
\usepackage{graphicx} % 处理图片
\usepackage{footnote} % 增强的脚注功能,可添加表格脚注
\usepackage{threeparttable} % 添加真正的表格脚注,示例见README
\usepackage{hyperref} % 添加pdf书签
\usepackage{psgo} % 添加围棋

\usepackage{tikz}
\usetikzlibrary{shapes,arrows,shadows}

% 字体设置
\setmainfont{Times New Roman}
\setsansfont[Scale=MatchLowercase,Mapping=tex-text]{PT Sans}
\setmonofont[Scale=MatchLowercase]{PT Mono}
\setCJKmainfont[ItalicFont={Kaiti SC}, BoldFont={Heiti SC}]{Songti SC}
\setCJKsansfont{Heiti SC}
\setCJKmonofont{Songti SC}
% \setCJKmainfont[BoldFont={FZXiaoBiaoSong-B05S}]{Songti SC}
% \setCJKfamilyfont{kai}[BoldFont=Heiti SC]{Kaiti SC}
% \setCJKfamilyfont{song}[BoldFont=Heiti SC]{Songti SC}
% \setCJKfamilyfont{hei}[BoldFont=Heiti SC]{Heiti SC}
% \setCJKfamilyfont{fsong}[BoldFont=Heiti SC]{Songti SC}
% \newcommand{\kai}[1]{{\CJKfamily{kai}#1}}
% \newcommand{\hei}[1]{{\CJKfamily{hei}#1}}
% \setromanfont[Mapping=tex-text]{TeXGyrePagella}
% \setsansfont[Scale=MatchLowercase,Mapping=tex-text]{TeXGyrePagella}
% \setmonofont[Scale=MatchLowercase]{Courier New}
%%设置常用中文字号,方便调用
\newcommand{\erhao}{\fontsize{22pt}{\baselineskip}\selectfont}
\newcommand{\xiaoerhao}{\fontsize{18pt}{\baselineskip}\selectfont}
\newcommand{\sanhao}{\fontsize{16pt}{\baselineskip}\selectfont}
\newcommand{\xiaosanhao}{\fontsize{15pt}{\baselineskip}\selectfont}
\newcommand{\sihao}{\fontsize{14pt}{\baselineskip}\selectfont}
\newcommand{\xiaosihao}{\fontsize{12pt}{\baselineskip}\selectfont}
\newcommand{\wuhao}{\fontsize{10.5pt}{\baselineskip}\selectfont}
\newcommand{\xiaowuhao}{\fontsize{9pt}{\baselineskip}\selectfont}
\newcommand{\liuhao}{\fontsize{7.5pt}{\baselineskip}\selectfont}

% 章节标题显示方式及页眉页脚设置
% \item xCJKnumb是自己额外安装的包
% \item titleformat命令定义标题的形式
% \item titlespacing定义标题距左、上、下的距离
\titleformat{\section}{\raggedright\large\bfseries}{\thesection}{1em}{}
\titleformat{\subsection}{\raggedright\normalsize\bfseries}{\thesubsection}{1em}{}
\titlespacing{\section}{0pt}{*0}{*2}
\titlespacing{\subsection}{0pt}{*0}{*1}
% 由于默认的2em缩进不够,所以我手动调整了,但是在windows下似乎2.2就差不多了,或者是article中没有这个问题
\setlength{\parindent}{2.2em}

% 设置表格标题前后间距
\setlength{\abovecaptionskip}{0pt}
\setlength{\belowcaptionskip}{0pt}

\renewcommand{\refname}{\bfseries{参~考~文~献}} %将Reference改为参考文献(用于 article)
% \renewcommand{\bibname}{参~考~文~献} %将bibiography改为参考文献(用于 book)
\renewcommand{\baselinestretch}{1.38} %设置行间距
\renewcommand{\figurename}{\small\ttfamily 图}
\renewcommand{\tablename}{\small\ttfamily 表}


\newcommand{\specialcell}[2][c]{%
  \begin{tabular}[#1]{@{}c@{}}#2\end{tabular}}


\title{通过Hoare逻辑来探索围棋的想法}
\author{苑明理}
\date{2017年7月}

\begin{document}

\maketitle{}
\renewcommand\contentsname{目录}
\setcounter{tocdepth}{2}
\tableofcontents

\newpage

\section{想法的产生}

围棋是一种复杂的棋类游戏,为了攻克它,人类把所有的认知工具都搬了过来,比如,棋感是一种直觉,棋招的计算是决策过程中对解空间的搜索。
最有趣的事情是,人们在这个过程中,创造了一系列的概念来辅助下棋,比如实地、外势、厚、薄等等。这些概念因为来自于直观,有它的模糊性。

最近 WeiqiTV 和 Deepmind 对 AlphaGo 的五盘自战棋做了详细解读,让人可以一窥它高深的棋力。在这个节目的解说中,樊麾提到 AlphaGo
对中腹有独到的认识。而常昊也说,人类能够计算清楚角和边,但算不清楚腹地。在下棋的过程中,人类往往会回避掉下在腹地的可能,因为计算困难,
妨碍了人们行棋的自由。

一个想法是:AI 能否和人类一起去探索围棋之道呢?它能否帮助人们创造性的提出一些围棋的新概念呢?这些概念可用来推理,并辅助人们来决策复杂问题。
这要求我们能够有一个基础来定义围棋里的概念,而这个让我想起了 Hoare 逻辑 。

\subsection{围棋机器的想法}

在引入 Hoare Logic 之前,在这一节首先让我们引入围棋机器的想法,我们会给出它的硬件和操作系统的描述。

\setcounter{gomove}{-1}
\begin{center}
\begin{psgoboard}
    \move*{q}{16}
    \move*{d}{16}
\end{psgoboard}
\end{center}

让我们从棋盘上格点的表示开始。如上,$\mathbf{q16}$ 和 $\mathbf{d16}$ 分别是黑方和白方所下的两个位置。整个棋盘有 361 个格点位置,
这对应于围棋机器的核心部件—盘面寄存器阵列。它由361个格点寄存器构成。每个格点寄存器的标号,就是相应格点在棋盘上面的标号。

每个格点寄存器有三种状态:$\mathit{black}$ 、 $\mathit{white}$ 和 $\mathit{unknown}$。

下棋的过程中,黑方和白方轮流进行,这对应于一个轮次寄存器 $\mathbf{turn}$,它有三种状态:$\mathit{black}$ 、 $\mathit{white}$、 $\mathit{judge}$。

围棋机器有指令寄存器,它的状态是指令集中的一条指令。指令集是由所有落子的可能构成,显然它们也是盘面上的所有位置,但我们用大写字母来标注它们。
另外,还有一条特殊的指令是 $\mathbf{PASS}$。

\setcounter{gomove}{-1}
\begin{center}
    \begin{psgoboard}
        \move*{q}{16}
        \move*{d}{16}
        \move*{r}{4}
        \move*{d}{4}
    \end{psgoboard}
\end{center}

围棋机器里有三个驻留内存的程序,它们具有类似的形式,但语义略有不同

\begin{itemize}
    \item 裁判子程序 $\mathbb{J}$:读入盘面寄存器的状态,给出一条指令,是期望去指定位置提子;裁判子程序可以写轮次寄存器,决定哪一方是下一手
    \item 黑方子程序 $\mathbb{B}$:读入盘面寄存器的状态,给出一条指令,是期望去指定位置下黑子
    \item 白方子程序 $\mathbb{W}$:读入盘面寄存器的状态,给出一条指令,是期望去指定位置下白子
\end{itemize}

围棋机器有一个非常简单的操作系统,它轮流执行上面三个子程序,根据轮次和指令,分别去提子、下黑子、下白子,或者报出错误。它对应于一个状态迁移表。

如上的开局作战,对应的指令执行序列是:$\mathbf{Q16; PASS; D16; PASS; R04; PASS; D04}$


\subsection{Hoare逻辑和程序规范}


\end{document}















